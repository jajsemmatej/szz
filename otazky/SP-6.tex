\subsection{SP-6 (DBS)}
Transakce a jejich vlastnosti - ACID.

\subsubsection*{Požadavky na databázi}
\begin{itemize}
	\item ochrana dat před chybami, havárií serveru --- řeší recovery modul (při systémových/hw chybách se DB vždy vrátí do konzistentního stavu)
	\item korektní, rychlý a asynchronní přístup pro větší počet uživatelů --- řeší concurency control modul (každý uživatel vidí konzistentní stav)
\end{itemize}

\subsubsection*{Transakce}
\begin{itemize}
	\item sekvence souvisejících akcí, která dostane DB z jednoho konzistentního stavu do druhého
	\item v průběhu může existovat nekonzistentní stav, ale vždy se musí dostat opět do konzistentního stavu
	\item tedy se provede buď celá, nebo vůbec
	\item začátek transakce --- vznik session nebo konec předchozí transakce
	\item konec transakce --- ukončení session, nebo klíčová slova COMMIT/ROLLBACK
	\item stavy transakce:
	\begin{itemize}
		\item aktivní (A) --- probíhají DML příkazy
		\item částečně potvrzená (PC) --- po provedení poslední transakce
		\item potvrzená (C --- Commited) --- po úspěšném zakončení (COMMIT)
		\item chybná (F) --- nelze pokračovat v normálním průběhu transakce
		\item zrušená (AB --- aborted) --- po skončení operace ROLLBACK
	\end{itemize}
\end{itemize}

\subsubsection*{ACID vlastnosti transakce}
\begin{itemize}
	\item Atomicity --- proběhne celá nebo vůbec
	\item Consistency --- transformuje DB z konzistentního stavu do konzistentního stavu
	\item Independence --- dílčí efekty jedné transakce nejsou viditelné jiným transakcím
	\item Durability --- uložené efekty jsou trvale uloženy
\end{itemize}

\subsubsection*{Recovery}
\begin{itemize}
	\item využívá se žurnál (log) obsahující změnové vektory
	\item různé typy chyb: globální (pád serveru, ztráta spojení, incident na disku) a lokální (logické chyby v konkrétní transakci)
	\item nedokončené transakce se odvolávají (ROLLBACK)
	\item potvrzené transakce, jejichž efekt se nestihl např. zapsat na disk se zopakují
\end{itemize}

\subsubsection*{Rozvrhování transakcí}
\begin{itemize}
	\item transakce a jejich dílčí operace se musí nějak rozplánovat
	\item toto rozvržení musí zajistit korektní asynchronní přístup pro více uživatelů
	\item pro snazší realizaci se používá zamykání a uzamykací protokoly
	\item legální rozvrh (nezaručuje uspořádatelnost):
	\begin{itemize}
		\item transakce musí mít objekt zamknutý, aby s ním mohla pracovat
		\item transakce nemůže zamykat již zamknuté objekty
	\end{itemize}
	\item tam se používají takzvané dobře formované transakce
	\begin{itemize}
		\item transakce zamyká objekt, chce-li k němu přistupovat
		\item transakce nezamyká objekt, pokud ho již zamkla
		\item transakce neodemyká objekt, který nezamkla
		\item na konci nezůstane žádný objekt zamčený
	\end{itemize}
	\item pro zamezení deadlocku se využívá dvoufázový uzamykací protokol
	\begin{itemize}
		\item 1. fáze --- pouze se uzamyká co je potřeba (klidně postupně spolu s dílčími operacemi transakcí, ale nesmí se nic odemykat)
		\item 2. fáze --- pouze se odemyká, po prvním odemčení se již nesmí nic zamykat
	\end{itemize}
	\item pokud jsou všechny transakce dobře formované a dvoufázové, lze pro ně vytvořit uspořádatelný rozvrh (aby na sebe nečekaly a nezamkly se)
	\item používá se rozvrhování i jiných operací, pak v případě deadlocku se u jedné operace prostě provede rollback
\end{itemize}