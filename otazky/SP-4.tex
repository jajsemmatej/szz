\subsection{SP-4 (AG1)}
Binární haldy, binomiální haldy. Binární vyhledávací stromy a jejich vyvažování. Hešovací tabulky.

\subsubsection*{Stromy}
\begin{itemize}
	\item zakořeněný strom --- jeden vrchol prohlášen za kořen, vzniká vazba mezi vrcholy předek-potomek
	\item binární strom --- zakořeněný strom, kdy každý vrchol může mít nejvýše 2 syny, a rozlišujeme, který je levý/pravý
	\item binomiální strom
	\begin{itemize}
		\item binomiální strom má vždy nějaký řád $k$
		\item obsahuje právě $2^k$ vrcholů
		\item nultý řád je jeden izolovaný vrchol
		\item další řády získáme tak, že vezmeme stromy všech předchozích řádů, a nad ně přidáme kořen
		\item to samé získáme tak, že vezmeme 2 binomiální stromy řádku $k-1$ a jeden připojíme pod kořen druhého
	\end{itemize}
\end{itemize}

\subsubsection*{Binární halda}
\begin{itemize}
	\item datová struktura tvaru binárního stromu
	\item v každém vrcholu je uložen klíč
	\item haldový tvar --- všechny hladiny kromě poslední musí být obsazené, poslední je obsazená zleva
	\item haldové uspořádání --- pro každou dvojici předek-potomek platí, že klíč v předku je menší nebo roven potomku
	\item základní operace:
	\begin{itemize}
		\item insert() --- dle tvaru haldy vloží na konec, pak probublá nahoru ($O(\log n)$)
		\item findMin() --- vrátí minimum, které je vždy v kořenu ($O(1)$)
		\item extractMin() --- odstraní z haldy minimum tak, že ho prohodí s posledním prvkem, smaže poslední prvek a zabublá kořen dolů ($O(\log n)$)
	\end{itemize}
	\item heapBuild() --- postavení haldy odspodu, každý list je korektní halda, postupně jdeme do vyšších hladin a nový kořen vždy probubláme dolů ($O(n)$)
	\item typicky se implementuje pomocí pole
\end{itemize}

\subsubsection*{Binomiální halda}
\begin{itemize}
	\item uspořádaná množina binomiálních stromů
	\item stromy jsou uspořádány vzestupně dle svých řádů
	\item celkový počet prvků v hladě je součet velikostí stromů
	\item pro každý řád $k$ se v haldě vyskytuje maximálně jeden binomiální strom
	\item vrcholy (podobně jako v binární haldě) obsahují klíče
	\item stromy si drží haldové uspořádání
	\item množina stromů se implementuje spojovým seznamem
	\item $n$-prvková halda má až $O(\log n)$ binomiálních stromů
	\item základní operace:
	\begin{itemize}
		\item findMin() --- vrátí minimum, které musí být v kořeni některého ze stromů (složitost $O(\log n)$, v případě implementace ukazatele na minimum je složitost $O(1)$)
		\item merge() --- sloučení 2 binomiálních hald
		\begin{itemize}
			\item postupuje se jako při binárním sčítání --- stromy se slučují od nejnižších řádů
			\item sloučení dvou binomiálních stromů je $O(1)$ --- stačí porovnat kořeny a zařadit větší pod menší
			\item složitost celého spojení je úměrná počtu stromů v haldách, tedy $O(\log n)$
		\end{itemize}
		\item insert() --- vytvoření jednoprvkové nové haldy, následně merge ($O(\log n)$)
		\item extractMin() --- nalezne se strom, kde se minimum nachází, tomuto stromu se utrhne kořen, a ze vzniklých stromů se vytvoří nová halda a provede se merge s původní ($O(\log n)$)
	\end{itemize}
\end{itemize}

\subsubsection*{Binární vyhledávací stromy}
\begin{itemize}
	\item binární stromy, kde v každém vrcholu je uložený unikátní klíč, a kde klíče v levém podstromu jsou vždy menší a v pravém větší než v aktuálním vrcholu
	\item $h(T)$ --- hloubka stromu
	\item operace:
	\begin{itemize}
		\item BVSShow() --- vypsání klíčů ve vzestupném pořadí ($O(n)$)
		\item BVSMin() --- nalezení nejmenšího prvku, je to ten nejvíc vlevo ($O(h(T))$)
		\item BVSMax() --- podobně jako Min()
		\item BVSPred() --- vrátí předchůdce prvku (ve vzestupném výpisu by byl levým sousedem) $O(h(T))$
		\begin{itemize}
			\item pokud má prvek levý podstrom, pak je předchůdce maximum z podstromu
			\item jinak je to první prvek, do kterého vstoupíme zprava při průchodu stromem nahoru
		\end{itemize}
		\item BVSSucc() --- následník, podobně jako Pred()
		\item BVSFind() --- hledá prvek s konkrétním klíčem ($O(h(T))$)
		\item BVSInsert() --- vloží prvek na správné místo ($O(h(T))$)
		\item BVSDelete() --- odstraní prvek ($O(h(T))$)
		\begin{itemize}
			\item pokud je prvek list, můžeme odtrhnout
			\item pokud má jednoho syna, nahradíme tímto synem
			\item pokud má 2 syny, prohodíme s následníkem/předchůdcem a smažeme potom (následník/před\-chůd\-ce existují a je v pravém/levém podstromu aktuálního vrcholu, a mají maximálně 1 syna)
		\end{itemize}
	\end{itemize}
	\item rychlost velké části operací závisí na hloubce stromu $h(T)$
	\item v nejhorším případě může být $h(T) == |T|$, tedy složitost operací by byla $O(n)$
	\item hloubku budeme regulovat vyvažováním
	\item lze vyvažovat "dokonale'' --- tedy rozdíl počtu prvků v levém a pravém podstromu každého prvku je maximálně 1, to ale není efektivní --- při jakékoliv implementaci bude složitost insert() nebo delete() $O(n)$
	\item proto využíváme hloubkově vyvážené stromy jako AVL
	\item AVL:
	\begin{itemize}
		\item rozdíl hloubek podstromů každého vrcholu může být maximálně 1
		\item hloubka celého BVS je pak $\Theta (\log n)$
		\item operace insert() a delete() musí udržovat hloubkové vyvážení (jiné ho nemění)
		\item v každém vrcholu se udržuje znaménko vyvážení
		\item pokud je více  než (mínus) jedna, je potřeba rotovat
		\item rotace se provede ve směru podle znaménka (podle nevyvážení stromu)
		\item vyvážení se opravuje jednoduchými nebo dvojitými rotacemi
	\end{itemize}
\end{itemize}

\subsubsection*{Hashovací tabulky}
\begin{itemize}
	\item skloubení nízkých paměťových nároků s efektivitou operací (pouze v průměrném případě)
	\item implementace slovníku --- klíč a hodnota, klíč je unikátní
	\item zvolíme konečné pole přihrádek a hashovací funkci $h$, která každému klíči univerza přiřadí jednu přihrádku
	\item pokud chceme vložit prvek s klíčem $k$ do tabulky, vložíme ho do přihrádky $h(k)$
	\item pokud budeme hledat prvek s klíčem $k$, víme že musí být v přihrádce $h(k)$
	\item protože ale univerzum klíčů bývá zásadně větší než počet přihrádek, stávají se kolize --- více klíčů se přiřadí do stejné přihrádky
	\item ideální hashovací funkce vypočte číslo přihrádky v konstantním čase a rozprostře zadanou $n$-prvkovou množinu rovnoměrně do všech $m$ přihrádek, tedy každá přihrádka max $\lceil n/m \rceil$ prvků
	\item příklady dobře fungujících hashovacích funkcí:
	\begin{itemize}
		\item lineární kongruence ($k \rightarrow a\cdot k \text{ mod } m$) --- $m$ je typicky prvočíslo, $a$ je vysoká konstanta nesoudělná s $m$
		\item vyšší bity součinu ($k \rightarrow \lfloor (ak \text{mod} 2^\omega)/2^{\omega - \mathcal{L}}$) --- hashujeme $\omega$-bitové klíče do $m = 2^\mathcal{L}$ přihrádek, vybereme $\omega$-bitovou lichou konstantu $a$, pak spočítáme $ak$, ořízneme na $\omega$ bitů a z nich vezmeme nejvyšších $\mathcal{L}$
		\item pro posloupnosti --- skalární součin nebo polynom
	\end{itemize}
	\item řešení kolizí
	\begin{itemize}
		\item chaining (řetězení) --- prvky se při kolizi vloží do stejné přihrádky do spojáku
		\item otevřená adresace --- klíč se při kolizi vloží na jiné. volné místo dle určitého pravidla:
		\begin{itemize}
			\item lineární přidávání --- prostě se posune o 1 vedle
			\item dvojité hashování --- znovu, novou hashovací funkcí se klíč přehashuje a uloží jinam ($h(k,i) = (f(k) + i \cdot g(k)) mod m$ --- $f,g$ jsou 2 různé hash funkce, $i$ značí kolikátý pokus to je na umístění)
			\item při mazání musíme označit, že tam nějaký prvek byl (aby se našly ty další co se daly jinam), říká se tomu náhrobek
		\end{itemize}
	\end{itemize}
	\item tabulka se může moc naplnit, pak ji chceme zvětšit, a klíče musíme přehashovat a umístit do rozšířené tabulky
\end{itemize}