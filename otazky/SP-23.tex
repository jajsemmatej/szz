\subsection{SP-23 (PA2)}
Abstraktní datový typ, jeho specifikace a implementace. Zásobník, fronta, pole, seznam, tabulka, množina. Implementace pomocí spojových struktur, stromů a pole.

\subsubsection*{Klasické datové typy}
\begin{itemize}
	\item specifikují množinu hodnot
	\item specifikují množinu operací s hodnotami
	\item mají jasně stanovenou implementaci, tedy vnitřní reprezentaci hodnot a realizaci operací
	\item dělíme na:
	\begin{itemize}
		\item jednoduché/primitivní --- nedělitelné/atomické hodnoty
		\item strukturované --- skládají se ze složek
		\item ukazatele --- adresy do paměti, závislé na typu na který ukazují
		\item jiné --- např. funkce
	\end{itemize}
\end{itemize}

\subsubsection*{Abstraktní Datový Typ (ADT)}
\begin{itemize}
	\item specifikuje množinu hodnot a množinu operací nezávisle na implementaci
	\item syntax je popsaná signaturou operace --- popis operace jako např. "\_ + \_ : int, int $\rightarrow$ int"
	\item sémantika je popsána pomocí axiomů, slovně nebo jinak
\end{itemize}

\subsubsection*{Zásobník (stack)}
\begin{itemize}
	\item kontejner, kde nejnovější prvky jsou odebrány první (LIFO)
	\item rozhraní: push(s, $x$), pop(s), top(s), isEmpty(s)
	\item implementace:
	\begin{itemize}
		\item pomocí pole --- má omezení, buď musí být konstantní velikost, nebo v případě nafukovacího může být push() lineární
		\item pomocí spojového seznamu
	\end{itemize}
\end{itemize}

\subsubsection*{Fronta (queue)}
\begin{itemize}
	\item kontejner, kde nejstarší prvky jsou odebrána první (FIFO)
	\item rozhraní: push(q, $x$), pop(q), front(q), isEmpty(q)
	\item implementace:
	\begin{itemize}
		\item pomocí pole --- má omezení, kromě problémy s velikostí je nutné pole "protočit" při operaci pop(), což je opět lineární 
		\item pomocí spojového seznamu
	\end{itemize}
\end{itemize}

\subsubsection*{Pole}
\begin{itemize}
	\item kontejner, kde lze ke všem prvkům přistupovat přímo v čase $O(1)$
	\item může být vícerozměrné (pak ale problémy s nafukováním)
	\item implementace:
	\begin{itemize}
		\item pomocí jednorozměrného pole (lze i více dimenzí, např. ukládat řádky za sebe)
		\item pomocí vícerozměrného pole 
	\end{itemize}
\end{itemize}

\subsubsection*{Seznam (list)}
\begin{itemize}
	\item kontejner, kde lze přistupovat ke všem prvkům
	\item implementace:
	\begin{itemize}
		\item (obousměrný) spojový seznam (pozor, přístup a vkládání jinam než na konec/konce seznamu je lineární)
	\end{itemize}
\end{itemize}

\subsubsection*{Množina (set)}
\begin{itemize}
	\item kontejner, který obsahuje prvky bez duplikátů
	\item implementace:
	\begin{itemize}
		\item seřazené/neseřazené pole --- fuj, bylo by to pomalý i na seřazeném
		\item pomocí spojového seznamu (seřazeného) --- fuj
		\item binární vyhledávací strom
		\item hash table
	\end{itemize}
\end{itemize}

\subsubsection*{Tabulka (asi mapa?)}
\begin{itemize}
	\item kontejner, který obsahuje data identifikovaná klíčem
	\item chová se jako množina, která má pod klíči navíc data
	\item implementace:
	\begin{itemize}
		\item pole (klíče == indexy v poli)
		\item pole (seřazené dle klíčů, obsahuje páry key-value)
		\item spoják (seřazený dle klíčů, obsahuje páry key-value)
		\item hash table
		\item binární vyhledávací strom
	\end{itemize}
\end{itemize}