\subsection{OB-21 (UKB)}
Hrozby síťové bezpečnosti, základní kategorie síťových útoků, principy DoS útoků (konkrétní příklady/techniky). Principy obrany proti síťovým útokům v moderních sítích.

\begin{itemize}
	\item je nutné zabezpečit všechny vrstvy (ISO/OSI model)
	\item základní rozdělení útoků:
	\begin{itemize}
		\item pasivní --- odposlech, analýza provozu
		\item aktivní --- MAC/IP spoofing, modifikace/přerušení komunikace
	\end{itemize}
	\item příklady útoků:
	\begin{itemize}
		\item ping of death --- DoS, poslání pingu (ICMP) tak, že sestavený síťový packet měl více než 64 KiB, OS si s tím dříve neuměly poradit $\Rightarrow$ buffer overflow
		\item ping flood --- DoS, zaplavení oběti pingy (ICMP Echo request), využíval pomalých linek připojení
		\item smurf attack --- DoS, útočník zaslal ping requesty s podvrženou zdrojovou adresou na různá zařízení (klidně broadcast), která pak polala reply na oběť --- amplification attack, malý payload útočníka, velké zahlcení oběti
		\item Man in the Middle --- útočník se vetře do komunikace mezi 2 uzly, díky tomu vidí a může měnit jejich komunikace (docílí se např. ARP poisoning, vystavením falešného AP, podvržením domény, napadením webových stránek...)
		\item SYN flood attack --- oběť má omezený počet TCP spojení, která může otevřít, útočník všechny vyčerpá tím, že naváže hodně spojení (počátkem TCP handshake --- zprávou SYN), oběť je ve stavu DoS
	\end{itemize}
	\item DoS (Denial of Service) obecně -- snaha vyřadit službu z provozu, lze provést např. zahlcením linky, vyčerpáním HW zdrojů oběti zneužitím špatně nastavené aplikace poskytující službu
	\item DDoS (Distributed DoS) --- DoS provedený z více míst najednou
	\item možnosti obrany:
	\begin{itemize}
		\item zamezení fyzického přístupu k síťovým prvkům, zamezení připojení "kamkoliv"
		\item segmentace sítě --- síťové oddělení např. různých oddělení firmy, každý vidí jen to co potřebuje
		\item firewall --- pravidla pro provoz, filtrování provozu, hlídání provozu, inspekce paketů
		\item používání šifrované komunikace
		\item proxy server --- kontroluje obsah komunikace
		\item DMZ (demilitarizovaná zóna) --- jediná část sítě přístupná přímo z internetu
		\item logy
		\item IDS/IPS
		\item omezení přístupu, autentizace a autorizace, nejmenší oprávnění...
	\end{itemize}
\end{itemize}