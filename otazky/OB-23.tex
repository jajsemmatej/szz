\subsection{OB-23 (ZSB)}
Digitální forenzní analýza, základní principy a procesy forenzní analýzy, digitální důkaz a digitální stopa, proces akvizice dat.

\begin{itemize}
	\item \textbf{Digitální forenzní věda} se zabývá zkoumáním a obnovením materiálu nalezeného na di\-gi\-tál\-ních za\-ří\-ze\-ních, často v kontextu kyberzločinů
	\item \textbf{forenzní analýza} je proces analyzování důkazů za účelem identifikace nebo rekonstrukce událostí, které způsobily tyto stopy
	\item \textbf{digitální forenzní analýza} je proces analyzování digitálních  důkazů za účelem identifikace nebo rekonstrukce událostí, které způsobily tyto digitální stopy
	\item jednodušeji DFA je proces odhalování a interpretace elektronických dat
	\item DFA se snaží zachovat důkazy v jejich nejoriginálnější podobě
\end{itemize}

\subsubsection*{Základní principy DFA}
\begin{itemize}
	\item Legalita 
	\item Integrita 
	\item Opakovatelnost (analýza nesmí mít vliv na kvalitu stopy)
	\item Možnost znovuprohlédnutí
	\item Nezávislost
	\item Expertíza
	\item Dokumentabilita
\end{itemize}

\subsubsection*{Procesy DFA}
\begin{itemize}
	\item role investigátora --- zjistit Kdo, Co, Kdy, Kde, Jak, Čím a Proč
	\item digitální důkaz --- jakákoliv důkazní informace přenášená/uložená v digitální podobě
	\item využití DFA --- dokazování kriminální aktivity, zkoumání bezpečnostních incidentů, ne-kriminální procesy
\end{itemize}

\subsubsection*{Vlastnosti digitálních důkazů}
\begin{itemize}
	\item nemateriálnost --- data jako taková jsou nehmotná, potřebují hmotné médium, které je sice součástí evidence, ale forenzní image dat se považují za originál
	\item latence digitální stopy --- digitální informace nelze spatřit pouhým okem, zároveň mohou navíc ukryty nebo zašifrovány
	\item časová identifikovatelnost --- digitální stopy často mívají značku, kdy byly vytvořeny
	\item obsahují mnoho informací
	\item problematická odolnost
	\item nízká hustota informací
	\item vysoká dynamika vývoje --- různorodé typy médií
	\item systémová dynamika --- data se často mění a přepisují
	\item složitá identifikace jedince --- je obtížné/nemožné s jistotou přiřadit člověka ke konkrétním akcím
	\item různé typy dat a platforem
	\item problém s určením fyzického úložiště dat --- např. cloud
	\item obrana přístupu --- šifrování (zašifrovaná data nemají význam)
	\item obnovitelnost --- některá zničená data lze obnovit
\end{itemize}

\subsubsection*{DFA --- obrazy disků}
\begin{itemize}
	\item obecně image/obraz reprezentuje obsah a strukturu logického disku či úložného zařízení
	\item image disků se využívají v DFA kvůli zachování dat --- speciální formáty ukládající metadata a kontrolní součty/hashe obsahu
	\item nejpoužívanější formáty jsou E01 a RAW/DD
\end{itemize}

\subsubsection*{Akvizice dat}
\begin{itemize}
	\item získání dat z fyzického přenašeče --- vytvoření forenzního image
	\item u image kontrola kontrolních součtů (kontrola integrity důkazu)
\end{itemize}