\subsection{OB-9 (BEK)}
Řízení přístupových oprávnění. Běh programu při nízkých oprávněních.

\subsubsection*{ACL}
\begin{itemize}
    \item ACL --- Access Control List, mocný nástroj zabezpečení, seznam ACE
    \item ACE --- Access Control Entry (jeden bezpečnostní záznam)
    \item SO --- Securable Object --- objekt, který může mít SD --- security descriptor
    \item DACL (Discretionary ACL) --- seznam ACE, které povolují/zakazují přístup k SO
    \item SACL --- seznam ACE, které definují typy přístupu k SO, které se logují
    \item Určení ACL --- zajistit prostředky, zjištění požadavků na zabezpečení, určení ACL technologie a převedení požadavků do ACL
\end{itemize}

\subsubsection*{Přístupové tokeny}
\begin{itemize}
    \item pro práci s oprávněními se ve windows používají přístupové tokeny
    \item popisují aktuální bezpečnostní kontext --- pravidla a atributy daého procesu/vlákna
    \item obsahují SID vlastníka, skupiny, uživatelského účtu, výchozí DACL, seznam práv uživatele, ...
    \item příklad bezpečnostního kontextu --- přihlášený uživatel, zadané heslo...
    
\end{itemize}

\subsubsection*{Běh s nejnižsím oprávněním}
\begin{itemize}
    \item každý program má dělat pouze k čemu byl určen a nic navíc
    \item má k tomu používat pouze nutná práva, ostatních se vzdát
    \item proč programy často potřebují zvýšit práva na administrátorská?
    \begin{itemize}
        \item ACL --- zápis do administrátorských složek
        \item problémy s právy --- pokud je právo opravdu nutné, holt s tím nic neuděláme, ale pokud ne, nesmíme ho vyžadovat
        \item přístup k LSA secrets
    \end{itemize}
    \item určení nezbytné úrovně práv:
    \begin{itemize}
        \item zjistit všechny prostředky používané programem
        \item zjistit všechny privilegovaná API, které program používá
        \item posoudit uživatelský účet, pod kterým má program běžet
        \item zjistit všechny práva tokenu
        \item určit nezbytná práva pro aplikaci
        \item změnit token --- zbavit se nepotřebných práv
    \end{itemize}
\end{itemize}

\subsubsection*{Windows UAC}
\begin{itemize}
    \item uživatel dostane po oprávnění token definující oprávnění
    \item administrátor dostane token s vysokými právy
    \item pro spuštění explorer.exe (a jeho potomků) se použije ořezaný token (ne admin práva)
    \item admin může svůj plný token využít pro spuštění jiných programů
    \item UAC kontroluje přístup, případně vyžádá administrátorské povolení
    \item mody UAC:
    \begin{itemize}
        \item Elevate without prompting
        \item promtp for consent
        \item prompt for credentials
    \end{itemize}
    \item UAC komponenty:
    \begin{itemize}
        \item Application Information Service (AIS) --- zajišťuje spouštění dalších aplikací s právy (vyššími)
        \item Consent.exe --- prezentuje uživateli UAC okno, výsledky předá AIS (je spouštěno AIS)
        \item virtuální ovladač (pro nekompatibilní aplikace)
    \end{itemize}

    \item Manifest --- vystavení, jaká práva program chce 

\end{itemize}