\subsection{SP-2 (AAG)}
Bezkontextové jazyky: Bezkontextové gramatiky, zásobníkové automaty a jejich varianty. Modely syntaktické analýzy bezkontextových jazyků.

\subsubsection*{Bezkontextové jazyky}
\begin{itemize}
	\item formální jazyk je bezkontextový právě tehdy, kdy jej lze generovat bezkontextovou gramatikou
	\item lze přijímat zásobníkovým automatem
	\item BJ jsou uzavřené vůči sjednocení, zřetězení a iteraci
	\item nejsou uzavřené pro průnik, doplněk a rozdíl
\end{itemize}

\subsubsection*{Bezkontextové gramatiky}
\begin{itemize}
	\item uspořádaná čtveřice: $G = (N, \Sigma, P, S)$, kde pravidla jsou ve tvaru: $A \rightarrow \alpha \quad (\alpha \in (N \cup \Sigma)^* A \in N)$
	\item bezkontextová gramatika gramatika může být zkracující
	\item nejednoznačná BG --- lze pro jeden řetězec z jazyka sestavit alespoň 2 různé derivační stromy
	\item jednoznační BG --- pro všechny řetězce existuje právě jeden derivační strom
	\item nejednoznačný bezkontextový jazyk --- neexistuje pro něj jednoznačná gramatika
	\item jednoznačný BJ --- lze generovat jednoznačnou gramatikou
	\item $\varepsilon$-pravidlo --- pravidlo tvaru $A \rightarrow \varepsilon$
	\item jednoduché pravidlo --- pravidlo tvaru $A \rightarrow B$
	\item gramatika je bez cyklů ($A \Rightarrow^+ A$), pokud neobsahuje jednoduchá a $\varepsilon$ pravidla
\end{itemize}

\subsubsection*{Zásobníkové automaty}
\begin{itemize}
	\item můžeme si představit jako konečný automat se zásobníkem
	\item (nedeterministický) ZA formálně: $R = (Q, \Sigma, G, \delta, q_0, Z_0, F)$
	\begin{itemize}
		\item $Q$ je konečná neprázdná množina stavů
		\item $\Sigma$ je konečná vstupní abeceda
		\item $G$ je konečná neprázdná abeceda zásobníku
		\item $\delta$ je přechodová funkce, definována jako zobrazení z $Q \times (\Sigma \cup \{\epsilon\}) \times G^*$ do množiny konečných podmnožin množiny $Q \times G^*$
		\item $q_0 \in Q$ je počáteční stav
		\item $Z_0 \in G$ je počáteční symbol na zásobníku
		\item $F \subset Q$ je množina koncových stavů
	\end{itemize}
	\item nedeterministický ZA je výpočetně silnější než deterministický
	\item DZA přijme řetězec, pokud přečte celý řetězec, a skončí buď v koncovém stavu, nebo s prázdným zásobníkem (výpočetně ekvivalentní způsoby, jdou na sebe převádět)
\end{itemize}

\subsubsection*{Syntaktická analýza}
\begin{itemize}
	\item levá derivace --- v každém kroku generování věty bezkontextovou gramatikou se nahradí neterminál, který je nejvíce vlevo
	\item pravá derivace --- v každém kroku generování věty bezkontextovou gramatikou se nahradí neterminál, který je nejvíce vpravo
	\item levý rozklad věty --- posloupnost čísel pravidel použitých v levé derivaci
	\item pravý rozklad věty --- posloupnost čísel pravidel použitých v pravé derivaci
	\item syntaktická analýza je proces, který pro danou bezkontextovou gramatiku $G$ a řetězec $\omega$ určí, zda $\omega \in L(G)$
	\item v kladném případě také získáme syntaktickou strukturu řetězce v podobě levého nebo pravého rozkladu
	\item metody syntaktické analýzy:
		\begin{itemize}
			\item shora dolů (top-down), která nalezne levý rozklad
			\item zdola nahoru (bottom-up), která nalezne pravý rozklad
		\end{itemize}
	\item jak pro danou gramatiku vytvořit ZA jako model syntaktického analyzátoru metodou shora dolů?
	\begin{itemize}
		\item 1 stav, přijímá prázdným zásobníkem
		\item provádí 2 operace --- expanzi a srovnání
		\begin{itemize}
			\item expanze: pokud mám na vrcholu zásobníku neterminál, vyndám ho a nahradím pravidlem gramatiky, ze vstupu nečtu nic
			\item srovnání: pokud mám na vrcholu zásobníku terminál, vyndám ho a zároveň přečtu ze vstupu
		\end{itemize}
		\item na začátku je na zásobníku počáteční neterminál
	\end{itemize}
	\item jak pro danou gramatiku vytvořit ZA jako model syntaktického analyzátoru metodou zdola nahoru?
	\begin{itemize}
		\item 2 stavy, přijímá přechodem do koncového stavu
		\item provádí 3 operace --- redukci, přesun a přijetí
		\begin{itemize}
			\item redukce: zůstávám ve stejném stavu, nečtu nic ze vstupu, pokud je na vrcholu zásobníku pravá strana pravidla, vyndám a nahradím příslušným neterminálem
			\item přesun: přečtu terminál ze vstupu a vložím na zásobník, zůstávám ve stejném stavu
			\item přijetí: na zásobníku je počáteční neterminál --- vyndám spolu s počátečním symbolem a přejdu do koncového stavu
		\end{itemize}
	\end{itemize}
\end{itemize}