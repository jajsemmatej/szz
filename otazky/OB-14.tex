\subsection{OB-14 (EHA)}
Etické hackování a penetrační testování. Metodologie provádění penetračních testů PTES a OWASP.

\subsubsection*{Penetrační testování (pentest(ing))}
\begin{itemize}
	\item definované, úzké zaměření
	\item typický cíl: počítačový systém, síťová služba, aplikace...
	\item identifikace bezpečnostních zranitelností, rizik a nespolehlivých prvků a prostředí
	\item výsledek je report/zpráva o výsledcích testu
	\item penetrační tester bývá často zaměřený na jednu oblast
\end{itemize}

\subsubsection*{Etické hackování}
\begin{itemize}
	\item široké zaměření
	\item všemožné hackerské metody objevování zranitelností
	\item pentest je podmnožina těchto činností
	\item Etický hacker má široké a hluboké znalosti, typicky nebývá specializovaný
\end{itemize}

Pojmy:
\begin{itemize}
	\item Black Hat Hacker
	\begin{itemize}
		\item různé, typicky škodlivé úmysly
		\item primární motivace bývá osobní/finanční zisk
		\item aktivity považovány za kyber-zločin
	\end{itemize}
	\item White Hat Hacker
	\begin{itemize}
		\item stejné metody jako Black Hat
		\item dobré úmysly, s povolením vlastníka systému --- tedy legální, zranitelnosti nahlášeny vlastníkovi
		\item motivace hlavně finanční odměna
	\end{itemize}
	\item Grey Hat aproach
	\begin{itemize}
		\item někdy legální, někdy ilegální aktivity
		\item přístup typicky závisí na cílovém systému
		\item nemívá škodlivé úmysly, ale občas porušuje zákony či etické standardy
	\end{itemize}
	\item Hrozba (Threat) --- událost/podmínka, která může způsobit škody nebo další následky
	\item Attack vector --- cesta/metoda umožňující škodlivou činnost či odhaluje zranitelnosti
	\item Attack Surface --- množina vektorů útoku
	\item Zranitelnost (Vulnerability) --- slabina, která může být zneužita (použitím attack vectoru), vystavuje systém hrozbě
	\item Zero-Day vulnerability --- nedávno nalezená zranitelnost, dříve neznámá
	\item Penetrační test --- povolená simulace útoku, výsledek je report
	\item různé metody testování:
	\begin{itemize}
		\item Black box --- tester má stejné informace jako běžný uživatel, žádná předchozí znalost systému
		\item White box --- tester má stejné informace a přístupy jako vývojový tým aplikace/systému
		\item Grey box --- podobné white box, ale informace jsou limitované
	\end{itemize}
	\item existují různé metodologie pro penetrační testování, jako PTES a OWASP
\end{itemize}

\subsubsection*{PTES metodologie}
\begin{itemize}
	\item Penetration Test Execution Standard
	\item návody, jak strukturovat pentest
	\item 7 fází pentestu
	\item Pre-engagement interactions
	\begin{itemize}
		\item domluví se rozsah/zaměření testu (jaké konkrétní aplikace/stroje/systémy/služby budou testovány)
		\item domluví se cena
		\item domluví se časový rozsah --- od kdy do kdy
		\item přijatelné praktiky sociálního inženýrství
		\item kontakty
		\item cíle klienta
		\item poskytnuté zdroje klientem
		\item pravidla vztahu --- oprávnění, testovaná verze, kdy se může testovat, odkud,...
	\end{itemize}
	\item Intelligence Gathering
	\begin{itemize}
		\item identifikace a pojmenování cíle
		\item získávání informací o cíli --- infrastruktura cíle, mapa služeb a systémů, chování zaměstnanců... jednoduše vše co se dá zjistit
	\end{itemize}
	\item Threat Modeling
	\begin{itemize}
		\item vysokoúrovňové modelování hrozeb
		\begin{itemize}
		
		\item analýza podnikového vlastnictví --- intelektuální vlastnictví, data zákazníků, kritičtí zaměstnanci
		\item analýza procesů --- technická i lidská infrastruktura
		\item analýza hrozeb ze strany lidí
		\item analýza možností hrozeb
		\end{itemize}
		\item model infrastruktury
		\begin{itemize}
		\item síťová topologie
		\item identifikace služeb
		\item mapování možných hrozeb a zranitelností
		\end{itemize}
		\item je potřebná kooperace klienta
	\end{itemize}
	\item Vulnerability analysis
	\begin{itemize}
		\item nejdůležitější (a nejcennější) jsou znalosti a zkušenosti testera
		\item hledání využitelných zranitelností
	\end{itemize}
	\item Exploitation
	\begin{itemize}
		\item na základě všech předchozích informací se ověřují zranitelnosti, tedy snaha využít domělé zranitelnosti
		\item jak univerzální nástroje, tak přímo nástroje ke zranitelnostem
		\item analýza zdrojového kódu/reverzní inženýrství
		\item fuzzing
	\end{itemize}
	\item Post Exploitation
	\begin{itemize}
		\item již jsme získali přístup
		\item ochránit data klienta
		\item ochránit sebe --- neprovést nic co není dohodnuté
		\item další analýza, co vše se dá zjistit / ukrást
	\end{itemize}
	\item Reporting
	\begin{itemize}
		\item vytvoření reportu, nejlépe na následující 2 části
		\item Executive summary --- vysokoúrovňová zpráva pro vedení, dopad na byznys, krátké shrnutí vážných nálezů
		\item Technical report --- podrobný popis celého procesu a postupu, popis nálezů, kroky pro replikaci a návrhy mitigace
	\end{itemize}
\end{itemize}

\subsubsection*{OWASP}
\begin{itemize}
	\item seznamy top 10 nejčastějších bezpečnostních problémů v různých kategoriích, např, webové aplikace
	\item lze použít pro vyzkoušení na testovaném cíli, a jako důkaz že se na nic důležitého nezapomnělo
\end{itemize}