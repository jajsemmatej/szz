\subsection{SP-3 (AG1)}
Základní pojmy teorie grafů. Grafové algoritmy: procházení grafu do šířky, určení souvislých komponent, topologické uspořádání, vzdálenosti v grafech, konstrukce minimální kostry a nejkratších cest v ohodnoceném grafu.

\subsubsection*{Základní pojmy}
\begin{itemize}
	\item neorientovaný graf --- uspořádaná dvojice $(V, E)$, kde $V$ je neprázdná konečná množina vrcholů a $E$ je množina hran
	\item hrana je dvouprvková podmnožina $V$, značíme $\{u,v\}$
	\item množina všech možných hran je $\binom{V}{2}$
	\item nechť $e = \{u, v\}$ je hrana grafu $G$, pak jsou $u,v$ koncové vrcholy / incidentní s $e$, a $u$ je sousedem $v$ a naopak 
	\item sled v grafu: sekvence vrcholů a hran v grafu dané délky (délka dána počtem hran), hrany musí být mezi navazujícími vrcholy (prostě jak a přes co se někam dostat) 
	\item cesta: sled, ve kterém se neopakují vrcholy (a tedy ani hrany)
	\item orientovaný graf --- dvojice $(V, E)$, kde $E$ je množina orientovaných hran
	\item stupeň vrcholu --- počet hran obsahujících vrchol (nebo také počet sousedů, pokud nepovolíme více hran mezi stejnými vrcholy)
	\item okolí vrcholu: množina sousedů
	\item regulární graf --- všechny vrcholy stejný stupeň
	\item úplný graf ($K_n$) --- množina vrcholů $E$ = $\binom{V}{2}$
	\item úplný bipartitní graf ($K_{m,n}$) --- tvořen 2 partitami, hrany vedou pouze z jedné do druhé
	\item cesta $P_m$
	\item kružnice $C_n$
	\item doplněk grafu $\overline{G}$ --- stejné vrcholy, ale doplněk hran (kde byly nejsou, kde nebyly jsou)
	\item podgraf --- množiny $V$ a $E$ jsou podmnožinami původního grafu
	\item indukovaný podgraf --- podgraf, jehož množina hran je maximální možná
	\item princip sudosti --- součet stupňů vrcholů je 2krát počet hran
	\item u orientovaných grafů je vstupní a výstupní stupeň vrcholu
	\item vstupní stupeň 0 --- zdroj
	\item výstupní stupeň 0 --- stok
	\item graf je souvislý, pokud mezi každými 2 vrcholy existuje cesta, jinak je nesouvislý
	\item souvislá komponenta --- maximální souvislý podgraf
	\item paměťová reprezentace grafu --- seznam sousedů ($O(|V| + |E|)$) nebo matice sousednosti ($O(|V|^2)$)
	\item symetrizace orientovaného grafu: pokud mezi vrcholy vede jakákoliv orientovaná hrana, nahradíme neorientovanou
	\item slabá souvislost --- symetrizace je souvislá
	\item silná souvislost --- pro každé dva vrcholy musí existovat cesta tam a zase zpátky 
	\item izomorfismus grafů --- $f(G)$ bijekce vrcholů na vrcholy, hrany musí být pořád ''mezi stejnými vrcholy''
	\item strom --- graf, který je souvislý a neobsahuje kružnici
	\item kostra grafu --- podgraf, který obsahuje všechny vrcholy původního grafu a je to strom
	\item topologické uspořádání orientovaného grafu --- seřazení vrcholů tak, aby hrany vždy vedly "zleva doprava" 
\end{itemize}

\subsubsection*{Grafové algoritmy}
\begin{itemize}
	\item BFS
	\begin{itemize}
		\item prohledávání grafu do šířky
		\item začne v daném vrcholu, postupně nově nalezené vrcholy přidává do fronty a z fronty postupně zpracovává a rozšiřuje dál
		\item lze použít pro hledání vzdálenosti mezi vrcholy (délka nejkratší cesty)
		\item také lze využít např. pro hledání souvislých komponent nebo kostry grafu
		\item při reprezentaci seznamem sousedů je časová i paměťová složitost $O(n + m)$
	\end{itemize}
	\item DFS
	\begin{itemize}
		\item prohledávání grafu do hloubky
		\item rozšiřuje v nejnověji nalezených vrcholech, lze použít zásobník místo fronty, nebo implementovat rekurzí
		\item lze použít pro hledání kostry
		\item také lze využít např. pro hledání souvislých komponent
		\item při reprezentaci seznamem sousedů je časová i paměťová složitost $O(n + m)$
	\end{itemize}
	\item TopSort
	\begin{itemize}
		\item výsledkem je topologické uspořádání vrcholů
		\item spočítá vstupní stupně vrcholů, najde zdroje (vstupní stupeň je nula) a postupně odebírá zdroje, tím vznikají vždy další (pokud b grafu není orientovaný cyklus)
		\item časová i paměťová složitost $O(n + m)$
	\end{itemize}
	\item Jarník
	\begin{itemize}
		\item výsledkem je minimální kostra ohodnoceného grafu
		\item začne v jednom vrcholu, postupně přidává další vrcholy podle toho, jaká je v aktuálním řezu grafu (mezi již vybranými vrcholy a ostatními) hrana s nejmenší hodnotou
		\item paměťová složitost $O(n + m)$, naivní implementace má časovou složitost $O(mn)$, implementace s použitím binárních minimových hald $O(m\log n)$
	\end{itemize}
	\item Kruskal
	\begin{itemize}
		\item výsledkem je minimální kostra ohodnoceného grafu
		\item začne se všemi vrcholy, ale bez hran
		\item postupně přidává hrany od nejlevnějších, zahazuje ty, které by vytvořily cyklus
		\item implementace s keříky (struktura Union-find) má složitost $O(m\log n)$
	\end{itemize}
	\item Dijkstra
	\begin{itemize}
		\item prohledávání ohodnoceného grafu (omezení na kladné ohodnocení hran)
		\item hledá nejkratší cesty
		\item podobně jako BFS prochází vrcholy, jde vždy ale do vrcholu, který je aktuálně nejblíže startovnímu vrcholu
		\item otevře nový vrchol (nejbližší startu), podívá se na hrany z něj a přepočítá případné rychlejší cesty do dosud neuzavřených sousedů
		\item pojem relaxace --- přepočítání sousedů vrcholu
		\item  složitost $O(m\log n)$ při implementaci s binární haldou
	\end{itemize}
	\item Bellman-Ford
	\begin{itemize}
		\item další relaxační algoritmus prohledávání ohodnoceného grafu (obecné ohodnocení hran, jen nesmí existovat záporné cykly)
		\item hledá nejkratší cesty
		\item podobně jako dijkstra, ale prochází vrcholy "náhodně" (ne podle vzdálenosti od startu, ale podle toho který byl dříve nalezený, podobně jako BFS)
		\item počítá se s tím, že uzavřené vrcholy lze znovu otevřít
		\item otevře nový vrchol, podívá se na hrany z něj a přepočítá případné rychlejší cesty do sousedních vrcholů, které mohou být i uzavřené, tedy je znovu otevře
		\item  složitost $O(m \cdot n)$
	\end{itemize}
\end{itemize}